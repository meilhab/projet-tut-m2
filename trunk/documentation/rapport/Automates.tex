\chapter{Les automates d'interface}

Ce sont Alfaro et Henzinger qui ont introduit la notion d'automates d'interface. Ils servent à modéliser les interfaces des composants et à décrire l'enchainement des appels de services. Ces automates sont issus des automates de type entrée/sortie où il n'est pas nécessaire d'avoir des actions d'entrée activables dans tous les états.\\

\section{Les modèles ou \enquote{sociable interfaces}}
Chaque composant est décrit par un seul automate d'interface. 
L'ensemble des actions est décomposé en trois ensembles : 
\begin{itemize}
\item les actions d'entrée qui sont les réceptions de messages et représentent les services offerts (identifiable par le caractère \enquote{?})
\item les actions de sortie qui sont l'envoi de messages et représentent les services requis (identifiable par le caractère \enquote{!})
\item et les actions internes qui sont des opérations locales (identifiable par le caractère \enquote{;}).\\

\end{itemize}

Le modèle des automates d'interface servira à la vérification de la compatibilité entre composants, sachant qu'un composant est muni de contrats spécifiques et de dépendances avec son environnement.

\section{Spécification de modèles}
Un modèle spécifie comment un composant interagit avec son environnement. C'est-à-dire qu'il décrit les hypothèses que le composant a en entrée sur son environnement, et garantit les hypothèses de sortie que le composant fournit.\\
Les modèles modélisent le comportement des entrées/sorties d'un composant par un automate.

\section{Composabilité entre modèles}
La composabilité est une condition impliquant les ensembles de variables et les actions d'un module, qui peuvent être vérifiés statiquement et efficacement. Essentiellement, deux
modules sont composables si il est logique d'envisager l'effet de leur communication.

\section{Compatibilité entre modèles}
Deux modèles sont compatibles s'il existe un environnement pour les assembler et s'assurer que les hypothèses des deux sont respectées.

