\documentclass{beamer}

\usepackage[T1]{fontenc}
\usepackage[latin1]{inputenc}
\usepackage[frenchb]{babel}
\usepackage{xcolor}
\usepackage{graphicx}
\usepackage{textpos}
\usepackage{url}

\usetheme{Warsaw}
%\usetheme{Antibes}

\setbeamertemplate{navigation symbols}{}

% Personnalisation du footer
\setbeamertemplate{footline}{
    \leavevmode%
    \hbox{\hspace*{-0.09cm}
    \begin{beamercolorbox}[wd=.4\paperwidth,ht=2.25ex,dp=1ex,center]{author in head/foot}
        \insertshorttitle
    \end{beamercolorbox}%
    \begin{beamercolorbox}[wd=.5\paperwidth,ht=2.25ex,dp=1ex,center]{section in head/foot}%
        \usebeamerfont{section in head/foot}\insertshortdate
    \end{beamercolorbox}%
    \begin{beamercolorbox}[wd=.1\paperwidth,ht=2.25ex,dp=1ex,right]{section in head/foot}%
        \insertframenumber{} / \inserttotalframenumber\hspace*{2ex}
    \end{beamercolorbox}}%
    \vskip0pt%
}


\hypersetup{
    pdfauthor   = {Benoit Meilhac, Catherine Almeida},%
    pdftitle    = {Transformation de diagramme de s�quences SysML vers les automates d'interfaces
        et v�rification de la compatibilit� avec l'outil Ticc},%
    pdfsubject  = {Soutenance de projet tuteur�},%
    pdfkeywords = {soutenance, transformation de mod�les, SysML, automate interface},%
    pdfcreator  = {PDFLaTeX},%
    pdfproducer = {PDFLaTeX}%
}

\title[Projet tuteur�]{
    Transformation de diagrammes de s�quences SysML vers les automates 
    d'interfaces et v�rification de la compatibilit� avec l'outil Ticc
}

\institute{
    D�partement informatique\\
    Universit� de Franche-Comt�\\
    Projet tuteur�\\
    Ann�e 2011-2012
}

\date{\today}

\author{
    Beno�t MEILHAC\\
    Catherine PINTO DOS SANTOS ALMEIDA
}

\begin{document}
	\begin{frame}
    	\titlepage
		\begin{textblock*}{2cm}(0cm,-2.5cm)
  			\includegraphics[scale=0.5]{./images/logoUfc.jpg}
		\end{textblock*}
		\begin{textblock*}{2cm}(8cm,-2.25cm)
  			\includegraphics[scale=0.5]{./images/logo_femto_st.jpg}
		\end{textblock*}

	\end{frame}


    \begin{frame}
    \frametitle{Plan}
    \tableofcontents%[hideallsubsections]

\end{frame}



    \section{Pr�sentation du sujet}

\subsection{Le sujet}
\begin{frame}{Le sujet}
    \begin{block}{Probl�matique}
        \begin{itemize}
            \item Mod�lisation graphique : SysML
            \item V�rification de l'assemblage de blocs
            \item N�cessit� d'un langage formel

        \end{itemize}

    \end{block}

\end{frame}

%

\subsection{SysML}
\begin{frame}{Mod�lisation de syst�mes}
	\centering
	\includegraphics[scale=0.5]{./images/sysml.jpg}

    \begin{block}{SysML}
        \begin{itemize}
        	\item System Modeling Language
            \item Langage de mod�lisation graphique
            \item Bas� sur UML
            \item Adapt� � l'Ing�nierie Syst�me (syst�mes complexes h�t�rog�nes)

        \end{itemize}

    \end{block}

\end{frame}

\begin{frame}{Points communs et divergences avec UML}
    \centering
    \includegraphics[scale=0.45]{./images/image5.jpg}

\end{frame}

%

\subsection{Automates d'interface}
\begin{frame}{Automates d'interface}
    \centering
    \includegraphics[scale=0.9]{./images/exempleAutomateInterface.jpg}
    \begin{block}{D�finition}
        \begin{itemize}
            \item Introduit par Alfaro et Henzinger
            \item Mod�lise les interface des composants
            \item Description des actions internes/entr�e/sortie

        \end{itemize}

    \end{block}

\end{frame}

%

\subsection{Vue globale du projet}
\begin{frame}{\'Etapes du projet}
    \centering
    \includegraphics[scale=0.45]{./images/vueEnsemble.png}

\end{frame}





    \section{Environnement Logiciel}

\subsection{Librairies et frameworks utilis�s}
\begin{frame}{Environnement Logiciel}
    \begin{block}{Elasticearch}
        \begin{itemize}
            \item Moteur d'indexation
            \item Puissant et distribu� (recherche en temps r�el)
            \item Int�gre deux moteurs de recherche : document et index
        \end{itemize}

    \end{block}
    
    \begin{block}{SableCC}
        \begin{itemize}
        	\item Analyse LALR (Look Ahead Lef Recursive)
            \item G�n�rateur de compilateur
            \item Parseur contruit automatiquement l'AST
            \item N{\oe}uds de l'AST strictement typ�s
            \item N{\oe}uds parcourus par un visiteur
            \item Visiteur utilise le Pattern visitor
        \end{itemize}

    \end{block}

\end{frame}

%

\subsection{D�marche de d�veloppement}
\begin{frame}{JUnit}
    \centering
    \includegraphics[scale=0.7]{./images/junit_logo.png}

    \begin{block}{}
        \begin{itemize}
            \item Test Driven Development
            \item Ecriture de tests unitaires
            \item Description des exigences du code
        \end{itemize}

    \end{block}

\end{frame}

    \chapter{R�alisation}
processus de validation de deux composants
buts <--
\section{M�ta mod�les SysML}
\subsection{M�ta mod�le du diagramme de s�quences}
\subsection{M�ta mod�le du diagramme de Ticc}
\section{Utilisation d'ATL}
\section{Parseurs XML vers Ticc}
\section{Compatibilit� des automates d'interfaces avec Ticc}

\clearpage


    \section{Bilan et Perspectives}
\begin{frame}{Bilan et perspectives}
	\begin{block}{Bilan}
		\begin{itemize}
            \item Utilisation de SysML
			\item Transformation de mod�les 
            \item Approche par composants (automates d'interface)
			\item Ouverture � la v�rification de la compatibilit� 
            dans l'assemblage de deux composants
		
		\end{itemize}
	
	\end{block}

	\begin{block}{Perspectives}
		\begin{itemize}
			\item Automatiser le processus avec un plugin Eclipse	
			\item Adapter un algorithme de parcours de BDD		
		
		\end{itemize}
	
	\end{block}

\end{frame}



\section*{}
\begin{frame}
	\centering
	\LARGE{D�monstration}

\end{frame}



\section*{}
\begin{frame}
	\centering
	\LARGE{Merci de votre attention.}

    \LARGE{Questions ?}

\end{frame}


\end{document}
