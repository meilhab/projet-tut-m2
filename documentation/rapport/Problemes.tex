\chapter{Probl�mes rencontr�s}

Durant le projet, divers probl�mes sont apparus. Plus ou moins complexes � r�soudre, ils n'ont n�anmoins pas emp�ch� la r�alisation de celui-ci.

Le premier probl�me fut de tenter de transposer l'algorithme fourni dans les r�gles ATL. Il est apparu que ce n'�tait pas possible du fait de la conception du langage ATL. Les r�gles propres aux m�ta mod�les ont donc �t� impl�ment�es.
Ensuite est apparu un probl�me technique concernant le site de Ticc~\cite{siteTicc}. %mettre la r�f�rence � la bibliographie
Le site n'�tait plus accessible, ce qui incluait donc plus de documentations le concernant. Par chance, le logiciel en lui-m�me avait d�j� �t� r�cup�r� pour pouvoir travailler dessus.

Suite � la transformation de mod�les, la transcription du fichier g�n�r� en syntaxe Ticc et l'ex�cution du code OCamL, s'est pos� le probl�me de savoir comment traiter le r�sultat obtenu. En effet, Ticc permet l'affichage de la compatibilit� de deux automates, cependant, cet affichage est une tentative d'assemblage des deux automates et la sortie g�n�r�e est cons�quente.

\clearpage
